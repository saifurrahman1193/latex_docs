%\documentclass[12pt]{article}
\documentclass[12pt]{book} % for chapters
%\usepackage{fullpage}
%\usepackage[top=1in,bottom=1in,left=1in,right=1in]{geometry}
\usepackage[margin=1in, paperwidth=8.5in, paperheight=11in]{geometry}
\usepackage[english]{babel}
\usepackage[utf8x]{inputenc}
\usepackage{amsmath}
\usepackage{graphicx}
\usepackage{caption}
\usepackage[depth=subsection]{bookmark}
\usepackage{multirow}
\usepackage[usestackEOL]{stackengine}
%https://www.sharelatex.com/learn/Code_listing
%For code view in latex

\usepackage{pdflscape}

\usepackage{listings}


%\usepackage[linktocpage=true]{hyperref}
\hypersetup{
    pdfborder = {0 0 0}
}

\usepackage[colorinlistoftodos]{todonotes}
\usepackage{chngcntr}
\usepackage{url}
%header and footer package
\usepackage{fancyhdr}
\usepackage[en-US]{datetime2}

%clearing default page heading and footer
\pagestyle{fancy}



\fancyhf{}
%
\ifx
     E for even page
    O for odd page
    L for left side
    C for centered
    R for right side 
\fi
\lfoot{\fancyfoot[LE,LO]{\leftmark}} 
\rfoot{ PAGE - \thepage} 



\fancyhead{}
\renewcommand{\footrulewidth}{1pt}
\renewcommand{\headrulewidth}{0pt}
\parindent 0ex
\renewcommand{\baselinestretch}{1.5}

% Redefine the plain page style
\fancypagestyle{plain}{%
    \lfoot{\fancyfoot[LE,LO]{\leftmark}} 
    \rfoot{ PAGE - \thepage} 
}




\begin{document}


\begin{titlepage}

\newcommand{\HRule}{\rule{\linewidth}{0.5mm}} % Defines a new command for the horizontal lines, change thickness here
\center % Center everything on the page
 
%----------------------------------------------------------------------------------------
%	HEADING SECTIONS
%----------------------------------------------------------------------------------------

\textsc{\Large Department of Computer Science \& Engineering}\\[0.5cm] % Name of your university/college
\textsc{\Huge United International University}\\[1cm] % Name of your university/college
 %\textsc{\Large Major Heading}\\[0.5cm] % Major heading such as course name
 % \textsc{\large Minor Heading}\\[0.5cm] % Minor heading such as course title

%----------------------------------------------------------------------------------------
%	TITLE SECTION
%----------------------------------------------------------------------------------------

\HRule \\[0.4cm]
{ \huge \LARGE \textbf{CSE6001: Advanced Database Systems}}\\ [0.4cm] % Title of your document
{ {\Large \textbf{Implementing a Food Delivery Service Database Using Object-Oriented Oracle and MongoDB}}}\\
\HRule \\ [1cm]
%  \textit{A dissertation submitted to the United International University in partial fulfillment of the
% requirement for the degree B. Sc. in Computer Science \& Engineering}
%----------------------------------------------------------------------------------------
%	AUTHOR SECTION
%----------------------------------------------------------------------------------------


\begin{minipage}{0.5\textwidth}
\begin{flushleft} \small
\emph{\textbf{\large Author}:}\\
Mohammad Saifur Rahman  \\ % Your name
ID : 0122420002\\


\end{flushleft}
\end{minipage}
~
\begin{minipage}{0.4\textwidth}
\begin{flushright} \large
\emph{\textbf{Supervisor}:} \\
\small
Dr. Mohammad Rezwanul Huq  \\% Supervisor's Name
Associate Professor \\
Department of CSE, EWU\\

\end{flushright}
\end{minipage}\\[3cm]

% If you don't want a supervisor, uncomment the two lines below and remove the section above
%\Large \emph{Author:}\\
%John \textsc{Smith}\\[3cm] % Your name

%----------------------------------------------------------------------------------------
%	DATE SECTION
%----------------------------------------------------------------------------------------

\ifx
\textregistered\textcopyright
\sffamily\textregistered\textcopyright
\fi

Copyright\textcopyright{Year \the\year}\\
%{\large \today}\\[2cm] % Date, change the \today to a set date if you want to be precise
\DTMlangsetup{showdayofmonth=false}
\today\\[1cm]
%----------------------------------------------------------------------------------------
%	LOGO SECTION
%----------------------------------------------------------------------------------------

\includegraphics[width=100px]{assets/uiu-logo.png}\\[1cm] % Include a department/university logo - this will require the graphicx package
 
%----------------------------------------------------------------------------------------

\vfill % Fill the rest of the page with whitespace

\end{titlepage}


%=========COVER PAGE ENDS HERE==========================
%=========COVER PAGE ENDS HERE==========================



\pagenumbering{roman}




%==================Table of contents PAGE starts HERE=============
%==================Table of contents PAGE starts HERE=============
%\frontmatter
\pagenumbering{arabic}
\counterwithin{equation}{chapter}
% \counterwithin{figure}{chapter}






% \thispagestyle{empty}
%     \addtocontents{toc}{\protect\thispagestyle{empty}}
% \tableofcontents
% \thispagestyle{empty}
% \clearpage

%\addtocontents{toc}{~\vfill}
\addtocontents{toc}{~\hfill{\large \textbf{Chapters}}}
\addtocontents{toc}{~\hfill{\large \textbf{Page}}\par}

%==================Table of contents PAGE ends HERE=============
%==================Table of contents PAGE ends HERE=============




\setcounter{page}{1}


%=================Introduction PAGE Starts HERE=================
%=================Introduction PAGE Starts HERE=================


\section*{Part 1: Implementing in Object-Oriented Oracle}
\section*{Part 2: Implementing in MongoDB}

Here I have used JSON schema to define the structure and validation rules for each collection. I have explicily used create commands. Though in mongodb we don\'t need to use create command. In each collection i have inserted at least 5 records or documents. Here I have used 

Here's how JSON schema is used in the code:
\begin{itemize}
    \item \textbf{schema:} Every collection has a validator that specifies JSON schema for that collection
    \item \textbf{data types:} Data types for each field, such as string, number, boolean, date, and array.
    \item \textbf{required fields:} It defines which fields are required and which are optional.
    \item \textbf{validation rules:} validation rules, such as regular expressions for email addresses or phone numbers, or range constraints for numerical values
\end{itemize}

MongoDB supports both reference-based and embedded document modeling,  but in this project I have used embedded documents to represent relationships between entities.




\begin{itemize}
    \item \textbf{food\_items collection:} here restaurant\_id: linking the food item to the restaurant collection.
    \item \textbf{oreders collection:} customer\_id linking the customer. food\_items: Array of ordered food items which is embedded, each with a food\_item\_id which is linking with food item collection. delivery\_personnel\_id linking the delivery persoannel collection. 
    \item \textbf{payments collection:} order\_id linking with order collection.
    \item \textbf{combo\_offers collection:} restaurant\_id linking with restaurant collection
\end{itemize}



%Introduction PAGE ends HERE----------
%Introduction PAGE ends HERE----------





\end{document}
