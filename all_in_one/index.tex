%\documentclass[12pt]{article}
\documentclass[12pt]{book} % for chapters
%\usepackage{fullpage}
%\usepackage[top=1in,bottom=1in,left=1in,right=1in]{geometry}
\usepackage[margin=1in, paperwidth=8.5in, paperheight=11in]{geometry}
\usepackage[english]{babel}
\usepackage[utf8x]{inputenc}
\usepackage{amsmath}
\usepackage{graphicx}
\usepackage{caption}
\usepackage[depth=subsection]{bookmark}
\usepackage{multirow}
\usepackage[usestackEOL]{stackengine}
%https://www.sharelatex.com/learn/Code_listing
%For code view in latex

\usepackage{pdflscape}

\usepackage{listings}


%\usepackage[linktocpage=true]{hyperref}
\hypersetup{
    pdfborder = {0 0 0}
}

\usepackage[colorinlistoftodos]{todonotes}
\usepackage{chngcntr}
\usepackage{url}
%header and footer package
\usepackage{fancyhdr}
\usepackage[en-US]{datetime2}

%clearing default page heading and footer
\pagestyle{fancy}



\fancyhf{}
%
\ifx
     E for even page
    O for odd page
    L for left side
    C for centered
    R for right side 
\fi
\lfoot{\fancyfoot[LE,LO]{\leftmark}} 
\rfoot{ PAGE - \thepage} 



\fancyhead{}
\renewcommand{\footrulewidth}{1pt}
\renewcommand{\headrulewidth}{0pt}
\parindent 0ex
\renewcommand{\baselinestretch}{1.5}

% Redefine the plain page style
\fancypagestyle{plain}{%
    \lfoot{\fancyfoot[LE,LO]{\leftmark}} 
    \rfoot{ PAGE - \thepage} 
}




\begin{document}


\begin{titlepage}

\newcommand{\HRule}{\rule{\linewidth}{0.5mm}} % Defines a new command for the horizontal lines, change thickness here
\center % Center everything on the page
 
%----------------------------------------------------------------------------------------
%	HEADING SECTIONS
%----------------------------------------------------------------------------------------

\textsc{\Large Department of Computer Science \& Engineering}\\[0.5cm] % Name of your university/college
\textsc{\Huge United International University}\\[1cm] % Name of your university/college
 %\textsc{\Large Major Heading}\\[0.5cm] % Major heading such as course name
 % \textsc{\large Minor Heading}\\[0.5cm] % Minor heading such as course title

%----------------------------------------------------------------------------------------
%	TITLE SECTION
%----------------------------------------------------------------------------------------

\HRule \\[0.4cm]
{ \huge \LARGE \textbf{CSE6011: Data Mining}}\\ [0.4cm] % Title of your document
{ {\Large \textbf{Forecasting SMS Traffic and Balance Availability with Machine Learning}}}\\
\HRule \\ [1cm]
%  \textit{A dissertation submitted to the United International University in partial fulfillment of the
% requirement for the degree B. Sc. in Computer Science \& Engineering}
%----------------------------------------------------------------------------------------
%	AUTHOR SECTION
%----------------------------------------------------------------------------------------


\begin{minipage}{0.5\textwidth}
\begin{flushleft} \small
\emph{\textbf{\large Author}:}\\
Mohammad Saifur Rahman  \\ % Your name
ID : 0122420002\\
Nirupom Das Dipto  \\ % Your name
ID : 0122230023\\
Md. Raqibur Rahman  \\ % Your name
ID : 0122310001\\
Md Mizanur Rahman \\ % Your name
ID : 0122420016\\

\end{flushleft}
\end{minipage}
~
\begin{minipage}{0.4\textwidth}
\begin{flushright} \large
\emph{\textbf{Supervisor}:} \\
Dr. Mohammad Nurul Huda  \\% Supervisor's Name
Professor, Head of CSE Dept. \\
Department of CSE, UIU\\

\end{flushright}
\end{minipage}\\[3cm]

% If you don't want a supervisor, uncomment the two lines below and remove the section above
%\Large \emph{Author:}\\
%John \textsc{Smith}\\[3cm] % Your name

%----------------------------------------------------------------------------------------
%	DATE SECTION
%----------------------------------------------------------------------------------------

\ifx
\textregistered\textcopyright
\sffamily\textregistered\textcopyright
\fi

Copyright\textcopyright{Year \the\year}\\
%{\large \today}\\[2cm] % Date, change the \today to a set date if you want to be precise
\DTMlangsetup{showdayofmonth=false}
\today\\[1cm]
%----------------------------------------------------------------------------------------
%	LOGO SECTION
%----------------------------------------------------------------------------------------

\includegraphics[width=100px]{assets/uiu-logo.png}\\[1cm] % Include a department/university logo - this will require the graphicx package
 
%----------------------------------------------------------------------------------------

\vfill % Fill the rest of the page with whitespace

\end{titlepage}


%=========COVER PAGE ENDS HERE==========================
%=========COVER PAGE ENDS HERE==========================



\pagenumbering{roman}




%==================Table of contents PAGE starts HERE=============
%==================Table of contents PAGE starts HERE=============
%\frontmatter
\pagenumbering{arabic}
\counterwithin{equation}{chapter}
% \counterwithin{figure}{chapter}






\thispagestyle{empty}
    \addtocontents{toc}{\protect\thispagestyle{empty}}
\tableofcontents
\thispagestyle{empty}
\clearpage

%\addtocontents{toc}{~\vfill}
\addtocontents{toc}{~\hfill{\large \textbf{Chapters}}}
\addtocontents{toc}{~\hfill{\large \textbf{Page}}\par}

%==================Table of contents PAGE ends HERE=============
%==================Table of contents PAGE ends HERE=============




\setcounter{page}{1}


%=================Introduction PAGE Starts HERE=================
%=================Introduction PAGE Starts HERE=================


\chapter{Introduction}


%Introduction PAGE ends HERE----------
%Introduction PAGE ends HERE----------




\section{Background}



\section{Literature Review}

\begin{itemize}
    \item \textbf{Literature:} work/writing of others
    \item \textbf{Review:}  anlalysis
    \item \textbf{Literature Review:}  anlalysis of the work or writing of others on a particular topics
    \item \textbf{3 parts:}  
    \begin{itemize}
        \item \textbf{Introduction:} 
            \begin{itemize}
                \item \textbf{Why is the topic worth reveiwing:} 
                \item \textbf{Research gap/problem statement:} 
                \item \textbf{Define key terms/concepts:} 
                \item \textbf{Overview of topics covered:} 
                \item \textbf{Criteria for organizations:} 
                \item \textbf{Scope of review:} 
            \end{itemize}
        \item \textbf{Body:}  
        \item \textbf{Conclusion:}  
    \end{itemize}
\end{itemize}



\section{Research Gap}
\section{Objectives}


% Section 1.1: Background

%     Overview of SMS usage and its significance.
%     Brief history of SMS technology.

%     Importance of accurate SMS traffic and balance prediction.

% Section 1.2: Literature Review

%     Existing research on SMS traffic prediction.
%     Review of machine learning techniques used for prediction.
%     Analysis of the limitations of previous studies.

% Section 1.3: Research Gap

%     Identification of the specific gap in the existing research.
%     Explanation of the need for a more comprehensive and accurate prediction model.

% Section 1.4: Objectives

%     Clear statement of the research objectives.
%     Outline of the goals to be achieved


    

\chapter{Methodology}
\section{Corpus Collection}
\section{Training Data}
\section{Test Data}

% Section 2.1: Corpus Collection

%     Description of the data sources used.
%     Explanation of the data collection process.
%     Data preprocessing techniques (cleaning,

%     normalization, etc.).

% Section 2.2: Training Data

%     Details of the training dataset.
%     Features selected for training.
%     Data splitting for training and validation.

% Section 2.3: Test Data

%     Description of the test dataset.
%     Evaluation metrics to be used.

% Section 2.4: Experimental Setup

%     Overview of the experimental design.
%     Configuration of the machine learning model.
%     Hyperparameter tuning methodology.


\chapter{Experimental Results and Analysis}
% Section 3.1: Model Performance

%     Presentation of the experimental results.

%     Evaluation of the model's accuracy, precision, recall, and F1-score.
%     Comparison with baseline models or previous studies.

% Section 3.2: Analysis of Findings

%     Discussion of the insights gained from the results.
%     Explanation of the model's strengths and limitations.
%     Identification of any unexpected outcomes.


\chapter{Conclusion and Future Works}
% Section 4.1: Summary of Findings

%     Recapitulation of the key research findings.

%     Highlights of the contributions made.

% Section 4.2: Limitations

%     Acknowledgment of the study's limitations.
%     Discussion of potential areas for improvement.

% Section 4.3: Future Directions

%     Suggestions for future research.
%     Potential extensions or applications of the work.

\end{document}
