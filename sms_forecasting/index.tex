%\documentclass[12pt]{article}
\documentclass[12pt]{book} % for chapters
%\usepackage{fullpage}
%\usepackage[top=1in,bottom=1in,left=1in,right=1in]{geometry}
\usepackage[margin=1in, paperwidth=8.5in, paperheight=11in]{geometry}
\usepackage[english]{babel}
\usepackage[utf8x]{inputenc}
\usepackage{amsmath}
\usepackage{graphicx}
\usepackage{caption}
\usepackage[depth=subsection]{bookmark}
\usepackage{multirow}
\usepackage[usestackEOL]{stackengine}
%https://www.sharelatex.com/learn/Code_listing
%For code view in latex

\usepackage{pdflscape}

\usepackage{listings}


%\usepackage[linktocpage=true]{hyperref}
\hypersetup{
    pdfborder = {0 0 0}
}

\usepackage[colorinlistoftodos]{todonotes}
\usepackage{chngcntr}
\usepackage{url}
%header and footer package
\usepackage{fancyhdr}
\usepackage[en-US]{datetime2}

%clearing default page heading and footer
\pagestyle{fancy}



\fancyhf{}
%
\ifx
     E for even page
    O for odd page
    L for left side
    C for centered
    R for right side 
\fi
\lfoot{\fancyfoot[LE,LO]{\leftmark}} 
\rfoot{ PAGE - \thepage} 



\fancyhead{}
\renewcommand{\footrulewidth}{1pt}
\renewcommand{\headrulewidth}{0pt}
\parindent 0ex
\renewcommand{\baselinestretch}{1.5}

% Redefine the plain page style
\fancypagestyle{plain}{%
    \lfoot{\fancyfoot[LE,LO]{\leftmark}} 
    \rfoot{ PAGE - \thepage} 
}




\begin{document}


\begin{titlepage}

\newcommand{\HRule}{\rule{\linewidth}{0.5mm}} % Defines a new command for the horizontal lines, change thickness here
\center % Center everything on the page
 
%----------------------------------------------------------------------------------------
%	HEADING SECTIONS
%----------------------------------------------------------------------------------------

\textsc{\Large Department of Computer Science \& Engineering}\\[0.5cm] % Name of your university/college
\textsc{\Huge United International University}\\[1cm] % Name of your university/college
 %\textsc{\Large Major Heading}\\[0.5cm] % Major heading such as course name
 % \textsc{\large Minor Heading}\\[0.5cm] % Minor heading such as course title

%----------------------------------------------------------------------------------------
%	TITLE SECTION
%----------------------------------------------------------------------------------------

\HRule \\[0.4cm]
{ \huge \LARGE \textbf{CSE6011: Data Mining}}\\ [0.4cm] % Title of your document
{ {\Large \textbf{Forecasting SMS Traffic and Balance Availability with Machine Learning}}}\\
\HRule \\ [1cm]
%  \textit{A dissertation submitted to the United International University in partial fulfillment of the
% requirement for the degree B. Sc. in Computer Science \& Engineering}
%----------------------------------------------------------------------------------------
%	AUTHOR SECTION
%----------------------------------------------------------------------------------------


\begin{minipage}{0.5\textwidth}
\begin{flushleft} \small
\emph{\textbf{\large Author}:}\\
Mohammad Saifur Rahman  \\ % Your name
ID : 0122420002\\
Nirupom Das Dipto  \\ % Your name
ID : 0122230023\\
Md. Raqibur Rahman  \\ % Your name
ID : 0122310001\\
Md Mizanur Rahman \\ % Your name
ID : 0122420016\\

\end{flushleft}
\end{minipage}
~
\begin{minipage}{0.4\textwidth}
\begin{flushright} \large
\emph{\textbf{Supervisor}:} \\
Dr. Mohammad Nurul Huda  \\% Supervisor's Name
Professor, Head of CSE Dept. \\
Department of CSE, UIU\\

\end{flushright}
\end{minipage}\\[3cm]

% If you don't want a supervisor, uncomment the two lines below and remove the section above
%\Large \emph{Author:}\\
%John \textsc{Smith}\\[3cm] % Your name

%----------------------------------------------------------------------------------------
%	DATE SECTION
%----------------------------------------------------------------------------------------

\ifx
\textregistered\textcopyright
\sffamily\textregistered\textcopyright
\fi

Copyright\textcopyright{Year \the\year}\\
%{\large \today}\\[2cm] % Date, change the \today to a set date if you want to be precise
\DTMlangsetup{showdayofmonth=false}
\today\\[1cm]
%----------------------------------------------------------------------------------------
%	LOGO SECTION
%----------------------------------------------------------------------------------------

\includegraphics[width=100px]{assets/uiu-logo.png}\\[1cm] % Include a department/university logo - this will require the graphicx package
 
%----------------------------------------------------------------------------------------

\vfill % Fill the rest of the page with whitespace

\end{titlepage}


%=========COVER PAGE ENDS HERE==========================
%=========COVER PAGE ENDS HERE==========================



\pagenumbering{roman}




%==================Table of contents PAGE starts HERE=============
%==================Table of contents PAGE starts HERE=============
%\frontmatter
\pagenumbering{arabic}
\counterwithin{equation}{chapter}
% \counterwithin{figure}{chapter}






\thispagestyle{empty}
    \addtocontents{toc}{\protect\thispagestyle{empty}}
\tableofcontents
\thispagestyle{empty}
\clearpage

%\addtocontents{toc}{~\vfill}
\addtocontents{toc}{~\hfill{\large \textbf{Chapters}}}
\addtocontents{toc}{~\hfill{\large \textbf{Page}}\par}

%==================Table of contents PAGE ends HERE=============
%==================Table of contents PAGE ends HERE=============




\setcounter{page}{1}


%=================Introduction PAGE Starts HERE=================
%=================Introduction PAGE Starts HERE=================


\chapter{Introduction}
This report presents a comprehensive analysis of SMS traffic and balance availability for a specific set of clients, with a particular focus on predicting potential service interruptions during holidays. By employing advanced machine learning techniques, such as \textbf{polynomial regression}, we aim to accurately forecast future SMS traffic and balance levels, enabling clients to optimize their usage and avoid unexpected service disruptions. 

By leveraging advanced data analytics, this study provides a comprehensive forecast of SMS traffic and balance levels, enabling organizations to proactively manage their client communication strategies and resource allocation. This predictive model accurately anticipates SMS traffic during peak periods, such as holidays, and offers hourly updates on balance levels for the next two days, ensuring a seamless and uninterrupted service experience for clients.

Ultimately, this research contributes to the field of machine learning applications and provides practical solutions for organizations seeking to effectively manage their SMS balance allocation to clients, particularly during peak demand periods like holidays.


%Introduction PAGE ends HERE----------
%Introduction PAGE ends HERE----------




\section{Background}
\subsection{The Problem}

In today's fast-paced world, SMS communication has become an integral part of our daily lives. For organizations with a large customer base, managing SMS traffic and ensuring adequate balance availability can be a complex challenge, especially during peak periods such as holidays. Service interruptions due to insufficient balance or overloaded networks can lead to customer dissatisfaction and financial losses.

\subsection{The Solution}

To address these challenges, this report presents a comprehensive analysis of SMS traffic and balance prediction \& automation for a specific set of clients. By employing advanced machine learning techniques, such as \textbf{polynomial regression}, we aim to develop a predictive model that can accurately forecast future SMS traffic and balance levels. This information will empower organizations to make informed decisions regarding their communication strategies and resource allocation, ensuring a seamless and uninterrupted service experience for their clients.

\subsection{The Benefits}
\begin{itemize}
		\item \textbf{Improved Service Quality:} Accurate predictions of SMS traffic and balance levels will enable organizations to proactively address potential issues, ensuring a consistent and reliable service experience for their clients.
        \item \textbf{Optimized Resource Allocation:}  By understanding future demand, organizations can allocate resources more efficiently, reducing employees hours of struggle during the holidays and minimizing service disruptions.
        \item \textbf{Enhanced Customer Satisfaction:} A reliable and uninterrupted SMS service can significantly improve customer satisfaction and loyalty.
        \item \textbf{Data-Driven Decision Making:} The insights gained from this analysis will provide organizations with a data-driven foundation for making informed decisions about their SMS communication strategies.
\end{itemize}


This report aims to contribute to the field of machine learning applications and provide practical solutions for organizations seeking to optimize their SMS communication infrastructure and deliver exceptional customer service.





\chapter{Literature Review}





\end{document}
